\documentclass{article}
\usepackage{amsmath}
\usepackage[cleanup, subfolder]{gnuplottex}
\usepackage[spanish, es-tabla]{babel}
\usepackage{booktabs}
\usepackage{xcolor}
\definecolor{tbcol}{rgb}{1,1,1}
\usepackage{amsmath}
\usepackage[left=2cm, right=2cm, top=2.5cm, bottom=3cm]{geometry}

\setlength{\parindent}{0pt}
\decimalpoint
\usepackage{Sweave}
\begin{document}
\input{analisis_daniel-concordance}

\textbf{Analisis de datos por estadística}\\
\textbf{Configuración 1}\\
La configuración 1 es: $L=172.8$ cm, $\delta L = 0.1$ cm, $m=105.50$ g, $\delta m = 0.05$ g

% latex table generated in R 4.4.2 by xtable 1.8-4 package
% Sun Dec  8 14:52:41 2024
\begin{table}[ht]
\centering
\begin{tabular}{rrrrrrr}
  \toprule
$n$ & $\nu$ (Hz) & $\delta\nu$ (Hz) & $\mu$ (kg/m$^3$) & $\delta\mu$ (kg/m$^3$) & $\bar{\mu}$ (kg/m$^3$) & $\sigma_{\bar{\mu}}$ (kg/m$^3$) \\ 
  \midrule
1 & 12.1 & 0.06 & 0.00059007 & 0.000003 &  &  \\ 
  2 & 23.9 & 0.10 & 0.00060498 & 0.000003 &  &  \\ 
  3 & 36.5 & 0.20 & 0.00058362 & 0.000003 &  &  \\ 
  4 & 49.3 & 0.20 & 0.00056872 & 0.000002 &  &  \\ 
  5 & 61.9 & 0.30 & 0.00056368 & 0.000003 & 0.00057616 & 0.000005 \\ 
  6 & 74.7 & 0.40 & 0.00055736 & 0.000003 &  &  \\ 
  7 & 85.6 & 0.40 & 0.00057773 & 0.000003 &  &  \\ 
  8 & 98.2 & 0.50 & 0.00057337 & 0.000003 &  &  \\ 
  9 & 111.2 & 0.60 & 0.00056591 & 0.000003 &  &  \\ 
   \bottomrule
\end{tabular}
\caption{Configuración 1} 
\end{table}

El valor reportado para la configuración 1 es:

$$\mu = \left(576\pm 5 \right)\times 10^{-6} \text{kg}/\text{m}$$

\textbf{Visualización de Datos}\\
A continuación se muestra en un gráfico de discrepancia la visaulización de los diferentes valores de $\mu$ comparado con el valor promedio obtenido:

\begin{figure}[!h]

\centering
\begin{gnuplot}[terminal=cairolatex]

set key top right
set grid
set title "Valores obtenidos de $\\mu$ vs numero de medida"
set xlabel "Número de medición"
set ylabel "Densidad lineal de masa $\\mu$"
set xtics 1
mu = 0.0005761598
set yrange [5.5e-4:6.2e-4]
plot[0:10] "mu_config1.dat" every :: 1 w errorbars t "Valores obtenidos de $\\mu$", \ 
mu t "Valor Promedio"

\end{gnuplot}
\caption{gráfico de discrepancia}

\end{figure}

\newpage

\textbf{Configuración 2}\\
La configuración 2 es: $L=148.7$ cm, $\delta L = 0.1$ cm, $m=126.1$ g, $\delta m = 0.05$ g

% latex table generated in R 4.4.2 by xtable 1.8-4 package
% Sun Dec  8 14:52:41 2024
\begin{table}[ht]
\centering
\begin{tabular}{rrrrrrr}
  \toprule
$n$ & $\nu$ (Hz) & $\delta\nu$ (Hz) & $\mu$ (kg/m$^3$) & $\delta\mu$ (kg/m$^3$) & $\bar{\mu}$ (kg/m$^3$) & $\sigma_{\bar{\mu}}$ (kg/m$^3$) \\ 
  \midrule
1 & 15.8 & 0.08 & 0.00055858 & 0.000003 &  &  \\ 
  2 & 33.4 & 0.20 & 0.00050000 & 0.000003 &  &  \\ 
  3 & 47.4 & 0.20 & 0.00055858 & 0.000002 &  &  \\ 
  4 & 63.1 & 0.30 & 0.00056036 & 0.000003 &  &  \\ 
  5 & 78.8 & 0.40 & 0.00056142 & 0.000003 & 0.00055500 & 0.000007 \\ 
  6 & 93.7 & 0.50 & 0.00057178 & 0.000003 &  &  \\ 
  7 & 109.4 & 0.50 & 0.00057091 & 0.000003 &  &  \\ 
  8 & 125.0 & 0.60 & 0.00057117 & 0.000003 &  &  \\ 
  9 & 143.7 & 0.70 & 0.00054698 & 0.000003 &  &  \\ 
  10 & 159.2 & 0.80 & 0.00055019 & 0.000003 &  &  \\ 
   \bottomrule
\end{tabular}
\caption{Configuración 2} 
\end{table}

El valor reportado para la configuración 2 es:

$$\mu = \left(555\pm 7 \right)\times 10^{-6} \text{kg}/\text{m}$$

\textbf{Visualización de Datos}\\
A continuación se muestra en un gráfico de discrepancia la visualización de los diferentes valores de $\mu$ comparado con el valor promedio obtenido:

\begin{figure}[!h]

\centering
\begin{gnuplot}[terminal=cairolatex]

set key top right
set grid
set title "Valores obtenidos de $\\mu$ vs numero de medida"
set xlabel "Número de medición"
set ylabel "Densidad lineal de masa $\\mu$"
set xtics 1
mu = 0.00055500
set yrange [5.0e-4:6.2e-4]
plot[0:11] "mu2.dat" every :: 1 w errorbars t "Valores obtenidos de $\\mu$", \ 
mu t "Valor Promedio"

\end{gnuplot}
\caption{gráfico de discrepancia 2}

\end{figure}

\end{document}


